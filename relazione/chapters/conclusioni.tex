\chapter{Conclusioni}
Abbiamo presentato le nostre implementazioni degli homework assegnati durante il corso, assieme ai risultati computazionali, riguardandi diverse strategie per la risoluzione del Problema del Commesso Viaggiatore. Abbiamo quindi riportato i risultati ottenuti testando le soluzioni implementate su alcune istanze della libreria TSPLIB, un benchmark standard per questo problema.

Il branch-and-bound costruito sul rilassamento lagrangiano basato su 1-albero si è rivelato efficace su istanze fino a circa 200-250 nodi; tuttavia, vi sono istanze di taglia inferiore a questo limite che non sono state risolte nel limite di tempo assegnato a ciascuna istanza. Per risolvere all’ottimo problemi di taglia superiore è stato necessario usare solver MIP, che ``ignorano’’ l’interpretazione reale del modello (ovvero il percorso di un agente attraverso tutti i nodi del grafo) e risolvono il modello di programmazione lineare mista-intera. Anche in questo caso è tuttavia necessario rilassare il problema eliminando i vincoli di eliminazione dei sottocicli, e inserirli come cutting planes quando necessario. Con questo approccio sono state attaccate con successo istanze fino a oltre 700 nodi. Infine abbiamo presentato alcuni approcci euristici di vario genere, sia motivati dall’interpretazione ``geometrica’’ del problema (NN e $k$-opt) sia basati su metodi di risoluzione ottima (branch-and-bound e solver MIP) per l’individuazione di buoni cicli hamiltoniani, anche subottimi, da impiegare anche come subroutine nei metodi precedenti per ottenere dei bound sul costo e delle soluzioni di partenza.

Pur ricordando che non necessariamente un’istanza ``grande’’ è più difficile di una di taglia più limitata, la chiave per approcciare istanze medio-grandi è ridurre drasticamente lo spazio di ricerca. Le prestazioni degli euristici e del subgradiente lagrangiano sono fondamentali in questo; nel caso questi forniscano dei bound laschi, lo spazio di ricerca non viene ridotto in maniera significativa e, dato lo stato di problema NP-completo, l’istanza rimane intrattabile. Ad esempio, la difficoltà nella risoluzione di \texttt{ts225}, che permane con la maggior parte dei metodi di risoluzione implementati, è dovuta alle pessime prestazioni del lagrangiano, che non riesce a calcolare un lower bound accettabile (l’upper bound invece è molto buono). In questo modo, non è possibile eliminare un numero di archi sufficiente a rendere l’istanza trattabile. È significativo che l’unico metodo tra quelli proposti in grado di trovare (senza certificare) la soluzione ottima sia il Polishing, che può esplorare in maniera più erratica lo spazio delle soluzioni, e che al contempo ha fornito risultati peggiori rispetto agli altri metodi su altre istanze.

Questi metodi, tuttavia, pur spinti al massimo grado possibile di ottimizzazione e tuning, sono sufficienti a risolvere istanze di difficoltà limitata. Per approcciare, ad esempio, istanze dell’ordine di migliaia di nodi è necessario adottare altre tecniche, motivo per cui la letteratura sul TSP è estremamente ricca. 
